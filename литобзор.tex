\documentclass[9pt]{article}
\usepackage{amsmath}
\usepackage{tikz}
\usepackage{pgfplots}
\RequirePackage{amsfonts}
\usepackage{enumitem}
\usepackage{epsfig}  
\usepackage[T2A,T1]{fontenc}
\usepackage[utf8]{inputenc}
\usepackage[russian,english]{babel}

\usepackage{tcolorbox}

\begin{document}
\title{\foreignlanguage{russian}{Обзор литературы}}
\author{\foreignlanguage{russian}{Перепелкин Владимир}}
\maketitle
\begin{otherlanguage*}{russian}
\subsection*{Обзор литературы}
\subsubsection*{Введение}
Во многих работах, где представлена многосекторная экономика и присутствует и внутриотраслевая, и межотраслевая несовершенная конкуренция, так или иначе, приходится описывать поведение фирм, и, тем самым, говорить о стратегиях, которых будут придерживаться фирмы в той или иной ситуации. Эти работы разделяются на те, где используется агентное моделирование и динамика системы выводится с помощью агентной симуляции, и где агентный подход не используется, и динамика системы выводится с помощью математического моделирования.
\subsubsection*{Математическое моделирование}
К моделям, где исследуется межотраслевое взаимодействие вне рамок агентного подхода, в первую очередь следует отнести работы  по межотраслевому балансу \footnote{ W. Leontief, Input-output economics, Oxford University Press, Oxford, 1986}\textsuperscript{,}\footnote{ Коссов В. В., Межотраслевой баланс, Экономика, М., 1966}. Эти модели описывают структуру экономики, никак не разделяя фирмы: есть лишь отрасли с определенными производственными технологиями, занимающиеся производством, фирм внутри отраслей нет, как и межотраслевой и внутриотраслевой конкуренции.

Важной предпосылкой моделирования экономики в терминах межотраслевого баланса является агрегирование, рассмотренное в работах А. А. Шананина\footnote{А. А. Петров, А. А. Шананин, “Экономические механизмы и задача агрегирования модели межотраслевого баланса”, Матем. моделирование, 5:9 (1993), 18–42}. Для того, чтобы описать экономику как состоящую из некоторого множества отраслей, необходимо правильное агрегирование экономических агентов, предприятий и продуктов -- важный этап в экономическом моделировании. В подобных работах выясняется, что возможности математического моделирования и выведения аналитических решений довольно сильно ограничены, ограничиваясь определенными видами функций и большим количеством экзогенно заданных процессов. 

Также можно вспомнить макроэкономическую модель RIM\footnote{Широв Александр Александрович, Янтовский Алексей Анатольевич Межотраслевая макроэкономическая модель rim - развитие инструментария в современных экономических условиях // Проблемы прогнозирования. 2017. №3. URL: https://cyberleninka.ru/article/n/mezhotraslevaya-makroekonomicheskaya-model-rim-razvitie-instrumentariya-v-sovremennyh-ekonomicheskih-usloviyah (дата обращения: 25.01.2024).}, показывающую структуру взаимодействий между отраслями и предназначенную для прогнозирования.
\subsubsection*{Агентное моделирование}
Закономерное затруднение при математическом моделировании межотраслевых экономических связей состоит в сложности, из-за чего приходится многие экономические явления предполагать экзогенно (спрос, ценообразование), а также агрегировать агентов, теряя микроэкономические обоснования и специфику взаимодействия внутри целого, в которое агенты были агрегированы. Агентное моделирование позволяет решить эти трудности, внедряя микроэкономические обоснования, ослабляя предпосылки на поведение агентов и вид функций.

В свою очередь, недостатком работ, использующих агентный подход, является то, что при описании поведения фирм многие моменты предполагаются экзогенно, и остается неясным, каково было бы поведение фирм, если бы такие предположения были ослаблены.
Так, например, в работах А.В. Леонидова и Е.Е. Серебряниковой по моделированию несовершенной конкуренции в многосекторной экономике\footnote{	А. В. Леонидов, Е. Е. Серебрянникова, “Исследование отклика на технологические шоки в многосекторной модели несовершенной конкуренции”, Пробл. управл., 2019, № 2, 30–40}\textsuperscript{,}\footnote{Леонидов Андрей Владимирович, Серебрянникова Екатерина Евгеньевна Динамическая модель несовершенной конкуренции в многосекторной экономике // Проблемы управления. 2017. №4. URL: https://cyberleninka.ru/article/n/dinamicheskaya-model-nesovershennoy-konkurentsii-v-mnogosektornoy-ekonomike (дата обращения: 25.01.2024).}
вид функции, которая определяет представления о кривой спроса (что необходимо для определения цен), задаётся экзогенно, причём предполагается, что сначала фирма производит, а потом уже назначает цены на произведенные товары: 
\begin{align*}
	\frac{Y_{it+1}}{Y_{it}} = \xi_{it} \cdot \left( \frac{P_{it+1} / P_t}{P_{it} / P_{t-1}}\right)^{-\eta}
\end{align*}
Параметр $\xi_{it}$ "оценивается производителем на основе реализовавшихся значений спроса и цен"\, а $\eta$ - экзогенно задаваемый параметр. Тот факт, что производители не знают вид "настоящей"\ кривой спроса и лишь пытаются оценить её, является достоинством такого подхода и приближает его к реальности\footnote{Bonanno G. General Equilibrium with Imperfect Competition // Journal of Economic Surveys. — 1990. — Vol. 4, N 4. — P. 297—328.}. Однако, с точки зрения исследования конкурентных стратегий фирм, такое предположение приводит к тому, что ограничиваются варианты ценовой политики фирм: фирма никак не использует информацию о поведении и состоянии других частных фирм, и, например, демпинг становится невозможным. Далее, в реальной жизни нет строгой последовательности между решениями об объёмах производства и решениями о ценах: фирмы не всегда принимают решения о ценах после производства товаров (иногда наоборот, иногда одновременно), из-за чего при моделировании конкурентных стратегий такая хронология могла бы сделать модель нереалистичной.

Другим аспектом данных работ, который оказывает значительное влияние на динамику экономики и поведение фирм, является наличие потребителей как отдельных экономических агентов, которые работают, получают заработную плату и долю прибыли компании, и осуществляют конечное потребление товаров. Потребитель определяет вектор спроса на конечную продукцию и предложение труда исходя из задачи максимизации функции ожидаемой полезности в условиях бюджетных ограничений, причём вид мгновенной функции полезности задаётся экзогенно:
$$u_{jt}=\sum_{i=1}^N \theta_{ji} \log C_{jit} - \psi_j L_{jt}^2, \quad \sum_{i=1}^N \theta_{jt} =1$$
В случае, когда целью ставится выяснение конкурентных стратегий фирм, подобное представление потребителей могло бы несколько усложнить анализ конкурентных стратегий. Поэтому может быть достаточно подхода, где работники рассматриваются как "агенты-фирмы"\, принимающие такие же решения, как и фирма, причём инвестиции в основной капитал могут быть проинтерпретированы как инвестиции в человеческий капитал, а инвестиции в оборотный капитал как затраты на воспроизводство рабочей силы (пищу, одежду, рекреационные ресурсы и т.д.).
\end{otherlanguage*} 
\end{document}
